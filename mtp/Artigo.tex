\documentclass[12pt]{article}

\usepackage{sbc-template}
\usepackage{graphicx,url}
\usepackage[brazil]{babel}   
\usepackage[utf8]{inputenc}  
\usepackage{float}



\sloppy

\title{CommunityLink: Plataforma Digital para Engajamento e Gestão de Voluntariado em Ações Comunitárias e Ambientais}

\author{Ananda Guedes do Ó\inst{1}}

\address{Instituto Federal da Paraíba (IFPB) -- João Pessoa - PB -- Brasil.\\
  \email{ananda.o@academico.ifpb.edu.br}}

\begin{document} 

\maketitle

\begin{abstract} 
CommunityLink is a digital platform designed to connect volunteers with community and environmental causes. It aims to facilitate engagement through an intuitive and accessible system, ensuring efficient volunteer management. The platform integrates technologies that enhance interaction between organizations and volunteers, optimizing participation and social impact. This article presents the theoretical foundations, development methodology, and technological structure supporting CommunityLink. Additionally, the article analyzes demographic data in conjunction with user preferences for digital volunteering platforms. Through the intersection of qualitative and quantitative research, patterns influencing volunteer participation, such as mobile device usage, desired functionalities, and corporate involvement, are identified. Based on this analysis, guidelines are proposed for the development of a mobile-first volunteering platform that meets user needs, promotes social interaction, and ensures clarity of information. 
\end{abstract}

\begin{resumo} 
CommunityLink é uma plataforma digital desenvolvida para conectar voluntários a causas comunitárias e ambientais. Seu objetivo é facilitar o engajamento por meio de um sistema intuitivo e acessível, garantindo uma gestão eficiente do voluntariado. A plataforma integra tecnologias que potencializam a interação entre organizações e voluntários, otimizando a participação e o impacto social. Este artigo apresenta as fundamentações teóricas, a metodologia de desenvolvimento e a estrutura tecnológica que sustentam o CommunityLink. Além disso, o artigo analisa dados demográficos juntamente com as preferências dos usuários em relação ao uso de plataformas digitais para voluntariado. Através do cruzamento de informações obtidas por meio de pesquisas qualitativas e quantitativas, foram identificados padrões que influenciam a participação em atividades voluntárias, como a utilização de dispositivos móveis, funcionalidades desejadas e a participação de empresas. Com base nesta análise, são propostas diretrizes para o desenvolvimento de uma plataforma de voluntariado mobile-first que atenda às necessidades dos usuários, promovendo a interação social e garantindo clareza nas informações. 
\end{resumo}

\section{Introdução}

Voluntariado desempenha um papel essencial no fortalecimento social e na preservação ambiental, promovendo o engajamento cívico e a colaboração entre indivíduos e organizações. No entanto, a gestão dessas iniciativas enfrenta desafios, como a dificuldade de recrutamento, a falta de comunicação eficiente entre voluntários e projetos, e a necessidade de otimizar processos administrativos. Nesse contexto, plataformas digitais emergem como ferramentas estratégicas para conectar voluntários a causas sociais de forma mais acessível e eficaz.

O \textit{CommunityLink} se propõe como uma solução inovadora para aprimorar esse processo, oferecendo um ambiente digital intuitivo e funcional que facilita a interação entre voluntários e organizações. A plataforma visa não apenas otimizar a gestão do voluntariado, mas também potencializar o impacto das ações comunitárias e ambientais.

Este estudo tem como objetivo apresentar a proposta do CommunityLink, explorando sua fundamentação teórica, a metodologia de desenvolvimento adotada e as tecnologias empregadas em sua construção. A pesquisa também analisa como a plataforma pode contribuir para a mobilização de voluntários, promover a transparência na gestão de projetos sociais e melhorar a experiência do usuário no engajamento digital.

Com o crescente uso de tecnologias digitais para facilitar a interação entre voluntários e organizações, as plataformas de voluntariado digital têm ganhado relevância. A criação de uma plataforma eficiente exige um entendimento profundo das preferências dos usuários em relação a funcionalidades, interface e interação social. Este estudo visa analisar essas preferências com base em questões relacionadas ao uso de tecnologia, funcionalidades desejadas e o papel das empresas no processo de voluntariado. A partir dessa análise, será possível identificar padrões que influenciam a participação e, assim, propor diretrizes para o desenvolvimento de uma plataforma digital que atenda às necessidades dos usuários e amplifique o impacto social.

\section{Fundamentação Teórica}
Os estudos publicados na Revista de Administração Mackenzie 22.6 (2021) abordam a crescente transformação digital e sua relevância no engajamento comunitário. A pesquisa enfatiza que a constante adaptação das plataformas digitais é essencial para manter sua eficácia e adesão dos usuários. No contexto do voluntariado, isso implica que as plataformas devem ser intuitivas, acessíveis e eficientes, garantindo que os voluntários se sintam motivados a participar de forma contínua. O ambiente digital não apenas facilita a participação, mas também cria novas oportunidades de interação, ampliando a mobilização e a conectividade das pessoas com as causas sociais. O estudo reforça a ideia de que, para aumentar a adesão e o impacto do voluntariado, as tecnologias precisam ser simples e funcionais, alinhadas às necessidades e expectativas dos usuários.

Da mesma forma, a pesquisa Equity and Volunteers. Legal Studies (2018) investiga as motivações e barreiras para a participação no voluntariado, com foco nos fatores sociais e psicológicos que influenciam a decisão de engajamento em ações comunitárias. O estudo destaca que, além da necessidade de plataformas acessíveis e funcionais, a motivação intrínseca dos voluntários — como a busca por um propósito ou o desejo de causar um impacto positivo — é um fator fundamental para o aumento da adesão. A pesquisa também sugere que as plataformas digitais podem desempenhar um papel essencial na conexão entre as necessidades da comunidade e as habilidades dos voluntários. No entanto, ressalta que a experiência do usuário deve ser adaptada para refletir essas motivações pessoais, criando um ambiente no qual os voluntários se sintam valorizados e engajados a longo prazo.

Além disso, a pesquisa Digital Platform for Social Innovation: Insights from Volunteering (2020) explora o uso de plataformas digitais para impulsionar a inovação social por meio do voluntariado. O estudo aponta que essas plataformas não apenas facilitam o engajamento de voluntários, mas também promovem novos modelos de colaboração e apoio entre organizações e indivíduos. A integração de tecnologias no voluntariado permite que iniciativas sociais ganhem visibilidade, otimizem processos e ampliem o impacto das ações. Além disso, a pesquisa destaca que o uso de ferramentas digitais possibilita o rastreamento e a mensuração do impacto das atividades de forma mais precisa, reforçando a transparência e incentivando mais pessoas a se envolverem, ao perceberem os resultados concretos de suas contribuições.

Embora abordem diferentes aspectos do voluntariado e do ambiente digital, esses três estudos convergem na importância da adaptação tecnológica para potencializar a eficácia das ações sociais. Eles enfatizam que a experiência do usuário, a acessibilidade das plataformas e a transparência nos resultados são fatores essenciais para fortalecer a participação e o engajamento dos voluntários. Ao integrar essas perspectivas, é possível compreender como o desenvolvimento de plataformas mais inclusivas e funcionais pode contribuir para a expansão do voluntariado e para a maximização do impacto social das iniciativas.

\section{Metodologia}
A metodologia adotada para o desenvolvimento da plataforma \textit{CommunityLink} seguiu um processo estruturado e iterativo, com o objetivo de atender de forma eficiente às necessidades dos usuários e otimizar a gestão do voluntariado. As etapas de desenvolvimento foram planejadas para garantir uma solução que fosse funcional, acessível e fácil de usar, priorizando a experiência do usuário e a integração de tecnologias adequadas. A seguir, detalham-se as principais fases do desenvolvimento, desde a coleta de requisitos até a prototipagem visual, que fundamentaram a construção da plataforma.

\subsection{Etapas de Desenvolvimento}

O desenvolvimento inicial da plataforma \textit{CommunityLink} seguiu um processo estruturado, priorizando a compreensão das necessidades dos usuários e a concepção de um protótipo visual. As etapas realizadas foram as seguintes:

\textbf{Levantamento de Requisitos:} Durante essa fase, foram analisadas as necessidades funcionais e não funcionais da plataforma, com foco em usabilidade, acessibilidade e eficiência na conexão entre voluntários e iniciativas. A análise foi fundamentada em pesquisas sobre plataformas similares e boas práticas de interação humano-computador (\textit{IHC}), garantindo que a solução fosse intuitiva e eficaz. A partir dessa análise, foram definidos os requisitos funcionais, casos de uso, diagramas de classes e diagramas de atividades, que auxiliaram na estruturação da plataforma.

\begin{figure}[H] 
\centering \includegraphics[width=0.8\textwidth]{imagens/casos_de_uso.jpeg} 
\caption{Diagrama de Casos de Uso} 
\label{fig:casos_uso}
\end{figure}

\begin{figure}[H] 
\centering \includegraphics[width=0.8\textwidth]{imagens/diagrama_classes.jpeg} 
\caption{Diagrama de Classes} 
\label{fig:diagrama_classes} 
\end{figure}

\begin{figure}[H] 
\centering \includegraphics[width=0.7\textwidth]{imagens/diagrama_atividades.jpeg} 
\caption{Diagrama de Atividades} 
\label{fig:diagrama_atividades} 
\end{figure}

\textbf{Coleta de Dados:} Para embasar o desenvolvimento, foi aplicado um formulário online direcionado a voluntários e organizadores de iniciativas sociais. O objetivo foi identificar os principais desafios enfrentados na organização do voluntariado e as expectativas em relação a uma plataforma digital. As informações coletadas orientaram a definição dos requisitos e das funcionalidades prioritárias do sistema.

\begin{figure}[H] 
\centering \includegraphics[width=0.8\textwidth]{imagens/forms_1.jpeg} 
\caption{Formulário aplicado para a coleta de dados} \label{fig:formulario} 
\end{figure}

\begin{figure}[H] 
\centering \includegraphics[width=0.7\textwidth]{imagens/forms_2.jpeg} 
\caption{Formulário - Dados Demográficos} 
\label{fig:formulario} 
\end{figure}

\begin{figure}[H] 
\centering \includegraphics[width=0.7\textwidth]{imagens/forms_3.jpeg} 
\caption{Formulário - Dados de Voluntariado} 
\label{fig:formulario} 
\end{figure}

\textbf{Análise de Dados:} A análise foi realizada por meio do cruzamento das respostas obtidas em diferentes pesquisas qualitativas e quantitativas. Cada resposta foi categorizada em tópicos que abordaram aspectos como tecnologias utilizadas, funcionalidades desejadas, participação de empresas, monitoramento de atividades, notificações e clareza das informações. A integração desses dados permitiu identificar padrões e convergências, oferecendo diretrizes para o desenvolvimento de uma plataforma mais eficiente e alinhada às expectativas dos usuários.

\textbf{Prototipagem Visual:} Com base na análise dos requisitos e nas respostas coletadas, foi desenvolvido um protótipo visual da plataforma. Este protótipo representa a estrutura da interface e o fluxo de navegação do sistema, permitindo validar a experiência do usuário antes da implementação da solução final.

\begin{figure}[H] 
\centering \includegraphics[width=0.8\textwidth]{imagens/tela_1.jpeg} 
\caption{Protótipo Visual - Ações Sugeridas} \label{fig:prototipo_visual} 
\end{figure}

\begin{figure}[H] 
\centering \includegraphics[width=0.8\textwidth]{imagens/tela_2.jpeg} 
\caption{Protótipo Visual - Seção de Depoimentos} 
\label{fig:prototipo_visual} 
\end{figure}

\begin{figure}[H] 
\centering \includegraphics[width=0.8\textwidth]{imagens/tela_3.jpeg} 
\caption{Protótipo Visual - ONGs Atuantes} \label{fig:prototipo_visual} 
\end{figure}

\subsection{Ferramentas e Tecnologias}

As ferramentas utilizadas nesta fase inicial do projeto foram:

\begin{itemize} \item \textbf{Google Forms:} Utilizado para a coleta de dados, permitindo mapear as necessidades e desafios enfrentados pelos usuários. \item \textbf{Figma:} Ferramenta empregada na criação do protótipo visual, possibilitando simular a interface da plataforma e testar fluxos de navegação. \item \textbf{Técnicas de Análise de Requisitos:} Aplicadas para definir as funcionalidades essenciais da plataforma, garantindo que o projeto atenda às demandas reais do público-alvo. \item \textbf{Diagramas UML:} Foram criados diagramas de classes e diagramas de atividades para estruturar a modelagem do sistema, assegurando coerência na implementação. \end{itemize}

Essa abordagem inicial permitiu a definição clara do escopo do projeto, garantindo que as futuras implementações estejam alinhadas às necessidades dos usuários e proporcionando uma base sólida para o desenvolvimento da plataforma.

\section{Resultados Esperados e Discussão}

Até o momento, o \textit{CommunityLink} atingiu marcos importantes em seu desenvolvimento, com a conclusão das etapas iniciais de levantamento de requisitos, coleta de dados e prototipagem visual. Essas fases foram fundamentais para definir a visão e os requisitos da plataforma, preparando o terreno para sua futura implementação.

Com o levantamento de requisitos, foi possível identificar as necessidades funcionais e não funcionais da plataforma, com foco especial em aspectos como usabilidade, acessibilidade e eficiência na conexão entre voluntários e organizações. Esse processo garantiu que a solução proposta seja intuitiva e eficaz, alinhando as expectativas dos usuários às melhores práticas de interação humano-computador. Além disso, foram desenvolvidos diagramas de classes e atividades, garantindo a organização estrutural do sistema e auxiliando na modelagem e implementação futuras.

A coleta de dados, realizada por meio de um formulário online, possibilitou mapear os desafios enfrentados por voluntários e organizadores de iniciativas sociais. Com isso, foi possível compreender melhor as principais dificuldades e expectativas em relação a uma plataforma digital, orientando a definição das funcionalidades essenciais a serem implementadas no futuro.

A prototipagem visual da plataforma representou um avanço significativo, oferecendo uma simulação da interface e do fluxo de navegação do sistema. Esse protótipo foi essencial para validar a experiência do usuário e assegurar que a interface proposta atendesse às necessidades e expectativas identificadas nas fases anteriores.

Esses resultados iniciais são promissores, pois estabeleceram uma base sólida para a futura implementação da plataforma \textit{CommunityLink}. A etapa de desenvolvimento está agora preparada para avançar com a construção do sistema, com um escopo bem definido e alinhado às demandas reais do público-alvo.

\subsection{Tecnologia como Facilitadora e Uso de Dispositivos}

A maioria dos participantes reconhece o valor da tecnologia na promoção de um maior engajamento com o voluntariado. A preferência por aplicativos móveis foi expressa de forma consistente, sendo os usuários da faixa etária de 26 a 40 anos os mais propensos a adotar soluções digitais móveis. Por outro lado, o grupo de 18 a 25 anos, embora altamente familiarizado com tecnologias, demonstrou baixa taxa de participação no voluntariado.

Dessa forma, é fundamental priorizar o desenvolvimento de uma solução mobile-first ou de um aplicativo dedicado, garantindo uma experiência de usuário otimizada, com notificações e navegação simplificada. A plataforma deve ser intuitiva e acessível, especialmente para usuários de diferentes faixas etárias, como os mais jovens, que têm familiaridade com a tecnologia, mas demonstram baixa participação em atividades voluntárias.

\subsection{Funcionalidades Desejadas}

Os participantes manifestaram interesse por funcionalidades que promovam a interação social, como fóruns e chats, além de ferramentas de filtro avançado, permitindo a personalização da busca por oportunidades de voluntariado. A análise também indicou que a clareza e transparência das informações são elementos cruciais para o sucesso da plataforma.

Nesse sentido, é importante implementar filtros avançados (por data, localização e causa), além de um histórico de atividades e mecanismos de feedback e avaliação. A plataforma deve incluir funcionalidades que promovam a interação entre voluntários e organizadores, permitindo que os usuários avaliem suas experiências e compartilhem suas opiniões.

\subsection{Participação de Empresas}

A participação de empresas foi vista como um diferencial positivo. Os participantes demonstraram interesse em ver empresas envolvidas, seja como patrocinadoras ou como parceiras de programas de voluntariado corporativo. Isso reflete um crescente desejo de ver a responsabilidade social corporativa mais integrada com ações de voluntariado.

Portanto, criar um espaço específico para empresas se registrarem e divulgarem suas iniciativas de voluntariado, além de incentivarem seus funcionários a participarem de forma ativa, seria uma medida importante. A plataforma deve promover parcerias estratégicas entre empresas e organizações de voluntariado, oferecendo visibilidade e reconhecimento institucional.

\subsection{Acompanhamento de Horas e Atividades}

A necessidade de monitorar horas de voluntariado foi destacada por alguns participantes, especialmente entre aqueles que já se engajam de forma regular nas atividades. No entanto, para os mais jovens, como estudantes e solteiros, o acompanhamento das horas não se mostrou uma prioridade.

Nesse caso, é recomendável oferecer a funcionalidade de acompanhamento de horas de forma opcional e personalizável. A plataforma deve permitir que os usuários interessados registrem e gerem relatórios ou certificados, ao mesmo tempo em que os outros usuários podem optar por não utilizar esse recurso.

\subsection{Notificações e Atualizações}

A implementação de notificações push foi amplamente solicitada pelos participantes, com o intuito de manter os usuários atualizados sobre novas oportunidades de voluntariado ou alterações nas atividades existentes. Esta funcionalidade é particularmente importante para os grupos de 26 a 40 anos, que possuem maior interesse em receber informações de forma dinâmica e em tempo real.

Por isso, a plataforma deve implementar notificações push para eventos de voluntariado, garantindo que os usuários sejam sempre informados sobre novas oportunidades, atualizações e mudanças nos eventos.

\subsection{Clareza e Objetividade nas Informações}

Os usuários destacaram que a clareza nas informações sobre eventos de voluntariado é crucial para o seu engajamento. As descrições devem ser detalhadas, com informações sobre horários, locais e impactos sociais.

Para garantir essa clareza, todas as informações dos eventos devem ser apresentadas de forma clara, objetiva e organizada, facilitando a compreensão e o acesso rápido aos detalhes das atividades. Isso é especialmente relevante para os usuários com nível superior, que podem ter expectativas mais altas quanto à transparência e qualidade das informações.


\section{Conclusão}

A transformação digital desempenha um papel crucial na modernização e otimização dos processos de voluntariado, oferecendo novas possibilidades para a gestão e engajamento de indivíduos em causas sociais e ambientais. Nesse contexto, a plataforma \textit{CommunityLink} surge como uma solução inovadora, com o objetivo de conectar de maneira eficiente voluntários e organizações. Utilizando tecnologias avançadas, a plataforma visa facilitar o gerenciamento das atividades voluntárias, proporcionando uma experiência intuitiva tanto para os voluntários quanto para as organizações.

Com o uso de ferramentas digitais, o \textit{CommunityLink} busca otimizar a comunicação, a transparência e o acompanhamento das ações de voluntariado, garantindo que as iniciativas sejam mais eficazes e atinjam um impacto social mais significativo. Além disso, ao integrar recursos que promovem a acessibilidade e a usabilidade, a plataforma tem o potencial de ampliar o alcance das causas sociais, incentivando a participação ativa de um maior número de pessoas e consolidando a colaboração entre a sociedade civil e as organizações.

A análise cruzada dos dados demográficos e das preferências dos usuários revela que a eficácia de uma plataforma de voluntariado depende da integração de diversos fatores, como a priorização de soluções mobile-first, a oferta de funcionalidades interativas e personalizáveis, e a participação ativa de empresas. A implementação de funcionalidades como filtros avançados, histórico de atividades e acompanhamento de horas de voluntariado são essenciais para melhorar o engajamento dos usuários. A clareza nas informações e a utilização de notificações push são práticas indispensáveis para manter o dinamismo da plataforma e garantir a satisfação dos voluntários.

Esses resultados iniciais demonstram que a plataforma \textit{CommunityLink} tem o potencial de se tornar uma ferramenta estratégica para potencializar o voluntariado, promovendo a inovação social e ampliando o impacto das ações comunitárias e ambientais. A implementação das diretrizes sugeridas contribuirá para garantir que a plataforma seja inclusiva, intuitiva e eficaz, ampliando seu alcance e impacto social.

\bibliographystyle{sbc}
\begin{thebibliography}{}

\noindent SCIELO HUMANAS.
\newblock "Transformação Digital: a constante necessidade de adaptação".
\newblock Blog SciELO Humanas, 22 dez. 2021.
\newblock Disponível em: \url{https://humanas.blog.scielo.org/blog/2021/12/22/transformacao-digital-a-constante-necessidade-de-adaptacao/}.

\vspace{0.2cm}

\noindent LE, G. H.; AARTSEN, M.
\newblock "Understanding volunteering intensity in older volunteers".
\newblock Ageing and Society, 12 out. 2022.
\newblock Disponível em: \url{https://www.cambridge.org/core/journals/ageing-and-society/article/understanding-volunteering-intensity-in-older-volunteers/50C99109202240BDC04E15630026EFBD}.
\newblock Acesso em: 20 dez. 2024.

\vspace{0.2cm}

\noindent CAMBRIDGE UNIVERSITY PRESS.
\newblock "Equity and volunteers".
\newblock Legal Studies, 2018.
\newblock Disponível em: \url{https://www.cambridge.org/core/journals/legal-studies/article/abs/equity-and-volunteers/136AD3C6C6B31FE4C5B03E4499090DDA}.

\vspace{0.2cm}

\noindent WILEY ONLINE LIBRARY.
\newblock "Digital platform for social innovation: Insights from volunteering".
\newblock Corporate Social Responsibility and Environmental Management, 2020.
\newblock Disponível em: \url{https://onlinelibrary.wiley.com/doi/full/10.1111/caim.12499}.

\vspace{0.2cm}

\noindent SCIELO BRASIL.
\newblock "O Voluntariado em questão: a subjetividade permitida".
\newblock Psicologia \& Sociedade, 2020.
\newblock Disponível em: \url{https://doi.org/10.1590/S1414-98932008000300003}.

\vspace{0.2cm}

\noindent IBICT.
\newblock "Informação, voluntariado e redes digitais".
\newblock Revista IBICT, 2019.
\newblock Disponível em: \url{http://ridi.ibict.br/handle/123456789/717}.

\vspace{0.2cm}

\noindent E-CAMPOS.
\newblock "A projeção de interesses em redes sociais de voluntariado".
\newblock E-CAMPOS Journal, 2021.
\newblock Disponível em: \url{https://e-compos.org.br/e-compos/article/view/162}.

\vspace{0.2cm}

\noindent REDALYC.
\newblock "As motivações no trabalho voluntário".
\newblock Revista Psicologia \& Ciência, 2015.
\newblock Disponível em: \url{https://www.redalyc.org/pdf/3885/388539113005.pdf}.

\vspace{0.2cm}

\noindent PEPSIC.
\newblock "Significado do trabalho voluntário empresarial".
\newblock Psicologia: Teoria e Pesquisa, 2015.
\newblock Disponível em: \url{https://pepsic.bvsalud.org/scielo.php?pid=S1984-66572015000200006&script=sci_arttext}.

\vspace{0.2cm}

\noindent CAMBRIDGE UNIVERSITY PRESS.
\newblock "Volunteers: An Essential Component".
\newblock Prehospital and Disaster Medicine, 2020.
\newblock Disponível em: \url{https://www.cambridge.org/core/journals/prehospital-and-disaster-medicine/article/volunteers-an-essential-component/8D0B3A75B36B385C66F9A5603E9961E7}.

\vspace{0.2cm}

\noindent REPOSITÓRIO UNINTER.
\newblock "Gamificação da disciplina metodologia da pesquisa no ensino superior: estudo de caso".
\newblock Google Acadêmico, 2018.
\newblock Disponível em: \url{https://repositorio.uninter.com/handle/1/330}.

\vspace{0.2cm}

\noindent ATADOS.
\newblock "Atados | Plataforma de Voluntariado | Encontre Trabalho Voluntário".
\newblock Atados, 2023.
\newblock Disponível em: \url{https://www.atados.com}.

\vspace{0.2cm}

\noindent V-VOLUNTEER.
\newblock "Volunteer | A plataforma para quem quer ajudar".
\newblock V-Volunteer, 2023.
\newblock Disponível em: \url{https://www.vvolunteer.com.br/}.

\vspace{0.2cm}

\noindent VOLUNTÁRIOS.
\newblock "Campanha Seja um Voluntário".
\newblock Voluntários, 2023.
\newblock Disponível em: \url{https://voluntarios.com.br/}.

\vspace{0.2cm}

\noindent IDEALIST.
\newblock "Idealist: Plataforma Global de Conexão do Setor Social".
\newblock Idealist, 2023.
\newblock Disponível em: \url{https://www.idealist.org/pt}.

\vspace{0.2cm}

\noindent LIBEROPINION.
\newblock "Voluntariado - Liberopinion - Plataforma de Participação Pública".
\newblock Liberopinion, 2023.
\newblock Disponível em: \url{https://liberopinion.com/voluntariado}.

\vspace{0.2cm}

\noindent TRANSFORMA BRASIL.
\newblock "Transforma Brasil".
\newblock Transforma Brasil, 2023.
\newblock Disponível em: \url{https://transformabrasil.com.br/}.

\vspace{0.2cm}

\noindent WORLD PACKERS.
\newblock "ONGs no Brasil: conheça projetos sociais e saiba como voluntariar".
\newblock Worldpackers, 2023.
\newblock Disponível em: \url{https://www.worldpackers.com/pt-BR/articles/ongs-no-brasil-voluntariado}.

\vspace{0.2cm}

\noindent ENTRAJUDA.
\newblock "Programas de Voluntariado da EntraAjuda".
\newblock EntraAjuda, 2023.
\newblock Disponível em: \url{https://www.entrajuda.pt/pages/introducao}.

\end{thebibliography}


\end{document}
